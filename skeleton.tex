%\documentclass{emulateapj}
\documentclass[twocolumn,apj,numberedappendix]{emulateapj}
\usepackage{hyperref}
\newcommand{\vdag}{(v)^\dagger}
\newcommand{\myemail}{tleung@astro.cornell.edu}
\newcommand{\Msun}{\mbox{$M_{\odot}$}}
\newcommand{\Rsun}{\mbox{R$_{\odot}$}}
\newcommand{\Lsun}{\mbox{L$_{\odot}$}}
\newcommand{\rarr}{$\rightarrow$}
\newcommand{\CO}{\mbox{CO($J$ = 3 $\rightarrow$ 2) }}
\newcommand{\Lp}{\mbox{$L^{\prime}_{\rm CO(1-0)}$}}
\newcommand{\LpU}{\mbox{K km s$^{-1}$ pc$^2$}}
\newcommand{\eg}{{\sl e.g.,~}}
\newcommand{\ie}{{\sl i.e.,~}}
\newcommand{\pmOne}{$^{-1}$}
\newcommand\tna{\,\tablenotemark{a}}
\newcommand\tnb{\,\tablenotemark{b}}
\newcommand\tnc{\,\tablenotemark{c}}
\newcommand\tnd{\,\tablenotemark{d}}
\newcommand\tne{\,\tablenotemark{e}}
\newcommand\tnf{\,\tablenotemark{f}}
\newcommand\tng{\,\tablenotemark{g}}

%\slugcomment{{\sc Accepted to ApJ:} August 1, 2006}
\usepackage{amsmath}
\usepackage{natbib}
\citestyle{aa}
\shorttitle{Study of a strongly lensed type-2 quasar host SMG at $z$ = 2.221}
\shortauthors{Leung \& Riechers}

\begin{document}
\title{A Massive Molecular Gas Reservoir in the Type-2 Quasar Host Galaxy SMM J0939+8315 at $z$ = 2.221  lensed by the Radio Galaxy 3C220.3}
\author{T. K. Daisy Leung and Dominik A. Riechers}
\affil{Department of Astronomy, Cornell University, Space Sciences Building, Ithaca, NY 14853, USA; tleung@astro.cornell.edu}

\begin{abstract}
We report the detection of \CO line emission in SMM J0939+8315 at $z$ = 2.221, a
strongly lensed high-redshift submillimeter galaxy (SMG) using
the Combined Array for Research in Millimeter-wave Astronomy (CARMA). The lensing system consists of a
foreground radio galaxy 3C220.3 and a companion galaxy at $z$ = 0.685. This allows us to place constraints on the intrinsic properties
of the cold gas and dust in the interstellar medium (ISM) of the background SMG which hosts a type-2 quasar. Prior to correcting for lensing 
amplification, we measure a velocity-integrated \CO line intensity of $I_{\rm CO(3-2)}$ = (10.7 $\pm$ 2.1) Jy km s\pmOne,
corresponding to a lensing-corrected CO($J$ = 1 \rarr 0) line luminosity of \Lp = (2.7 $\pm$ 
blah) $\times$ 10$^{10}$ \LpU. This
translates to a molecular gas mass of $M_{\rm gas}$ = 2.2 $\times$ 10$^{10}M_\odot$. We report marginally resolved continuum 
emission from the foreground galaxy with peak flux density of $S_\nu$ = 5.56 $\pm$ 0.54 mJy
 at 104 GHz, allowing us to put constraints on the spectral energy distribution (SED) of this galaxy. We 
fit
 both an optically thick and optically thin modified blackbody models to the SED of the SMG, the preferable optically thick model  yields dust mass of $M_{\rm
dust}$ = 50.5$^{20.4}_{-20.2}$ $\times$ 10$^8$\Msun, and total infrared (IR) luminosity of $L$ = 88.5$^{+2.6}
_{-2.6}\times$10$^{12}$\Lsun, prior to lensing correction. We conclude that the properties (\eg gas mass, gas mass 
fraction, SFR) of the molecular gas reservoir in SMM
J0939+8315 based on our \CO observation is similar to other high redshift
SMGs. \\
NEED TO  FIX: 
report SFR, gas-to-dust ratio.. Comparison, etc. 
\end{abstract}
\keywords{galaxies: formation --- galaxies: high-redshift --- submillimeter: galaxies}

\section{Introduction}\label{sec:intro}
Submillimeter-selected galaxies (SMGs) are predominantly found at redshift $z \sim$ 1$-$3, during the epoch of stellar mass and 
galaxy assembly. These galaxies are luminous in submillimeter (submm) due to the re-radiation from the dust components in the
 far-infrared (FIR) wavelengths. Followup observations of SMGs discovered in large sky surveys (\eg HerMES, H-ATLAS) (cite) with sub-(mm) facilities have led to growing consensus
  of the observable properties of SMGs (cite, blah). SMG is a population of high-redshift galaxy that are typically dusty, gas-rich, 
  and luminous ($\gtrsim$ 10$^{12}$ \Lsun) in infrared with high star formation rates ($\gtrsim $ 500 \Msun yr\pmOne) \citep[\eg][]{Lagache05a}.
  
  To characterize the physical properties of the gas reservoirs in the ISM where active star formation takes place, carbon monoxide (CO) rotational lines have been commonly used as tracers due to the high abundance of this molecule in the ISM as well as its low excitation energy, the ground state transition line thereby directly probes the cool gas that is essential to fuel star formation \citep{Carilli13a}. Recent observations of the CO line of SMGs at $z \sim$ 1$-$3 have  
provided evidence that SMGs have large gas reservoirs typical of \textgreater 10$^{10}$\Msun. In most cases, detailed studies are carried out on SMGs that are gravitationally lensed as lensing amplifies the intrinsic luminosities of these sources, making them the brightest unveiled in large sky surveys \citep{Negrello10a,Vieira10a}. \par
One of such lens system has a peculiar configuration consisting of a SMG, SMM J0939+8315 (hereafter SMM J0939), hosting a 
type-2 quasar at $z$ = 2.221. SMM J0939 is being strongly lensed by a double-lobed Fanaroff-Riley Class II (FR-II) \citep*{Fanaroff74} radio galaxy (3C220.3) which has a 
companion galaxy B at $z$ = 0.685, separated by 1\farcs5. SMM J0939 is currently one of the brightest known lensed
SMGs, with lensing-magnified flux density of SMM J0939 $F_{\rm 250\mu m}$ = 440 $\pm$ 15 mJy. The redshift of SMM J0939 has 
been measured through the detection of CIV 1459\AA\
 and HeII 1640\AA\
line emissions \citep[hereafter H14]{Haas14}. These line detections provide conceivable evidence of active galactic nuclei (AGN) activity in this SMG, suggesting the presence of a type-2 quasar. 

In this paper, we present the detection of \CO line emission in the background SMG
SMM J0939 at $z$ = 2.221 obtained with the Combined
Array for Research in Millimeter Astronomy (CARMA), and the study of the ISM in great detail. The underlying continuum at 104 GHz allows us to put constraints on the SED of the 
foreground FR-II galaxy at mm wavelengths. 

This paper is organized as follows: in \S \ref{sec:obs}, we describe the
observations of the \CO line emission with CARMA; in \S \ref{sec:res}, we report the
detection of CO emission in the background galaxy, as well as continuum emission in the foreground galaxy; in \S
\ref{sec:Lens}, we present our lens modeling analysis; in \S \ref{sec:SED}, we perform SED fitting to 3C220.3
and SMM J0939, and derive the intrinsic properties of the interstellar medium (ISM) in the SMG; in \S \ref{sec:conclusions}, we
conclude our findings by comparing to other similarly bright, strongly-lensed SMGs at similar redshifts.

We adopt a standard $\Lambda$CDM cosmological model throughout this paper, with H$_0$= 69.32 km Mpc\pmOne
s\pmOne, $\Omega_{\rm M}$ = 0.286, $\Omega_\Lambda$=0.713, based on WMAP9 results \citep{Hinshaw13a}.
The luminosity distances at $z$ = 0.685 and $z$ = 2.221 are 4214 Mpc and 19052 Mpc, respectively; 1$\arcsec$
corresponds to 8.406 kpc at $z$ = 2.221, and 7.169 kpc at $z$ = 0.685.

\section{Observations}\label{sec:obs}
\subsection{CARMA} \label{sec:carmadata}
Observations of the \CO rotational transition ($\nu_{\rm rest}$ = 345.7959899 GHz) toward the background galaxy SMM
J0939 ($z$ = 2.221) were carried out using CARMA at the redshifted frequency of $\nu_{obs}$ = 107.357 GHz (2.79 mm)  (program ID: cf0142; P.I.: Riechers). The 3 mm receivers were used to cover the redshifted CO($J$ = 3 $\rightarrow$ 2) line and the nearby 2.88 mm continuum emission, employing a correlator setup providing a bandwidth of 3.75 GHz in each sideband and spectral resolution of 5.208 MHz ($\sim$14.5 km s\pmOne). The line was placed in the
upper sideband with the local oscillator tuned to $\nu_{\rm LO}\sim$104.2609 GHz.
Observations were carried out under good
weather conditions in the E array configuration on 2014 July 12. This resulted in 1.56 hours of 15 antenna-
equivalent on-source time after discarding unusable visibility data.
The nearby source J1039+811 (0.65Jy) was observed every 20 minutes for
pointing, amplitude, and phase calibration. Mars was observed as the primary
absolute flux calibrator, and the quasar 3C273 was observed as the secondary
flux calibrator. J0927+390 was observed for bandpass calibration, yielding $\sim
$15\% calibration accuracy.
The {\sc MIRIAD} package was used to calibrate and analyze the visibility data which are deconvolved and imaged using
the CLEAN algorithm with ``natural" weighting. This yields a synthesized clean beam size of 13$\farcs$6 $\times$ 
5\farcs8 for the upper sideband image cube. The final rms noise is $\sigma_{\rm ch}$ = 13.7 mJy beam\pmOne\ per bin with a channel width of $\sim$29 km\pmOne, and $\sigma$ = 1.825 Jy km s\pmOne\ $\rm beam^{-1}$ over a channel width of 218.7 MHz (corresponding to $\sim$610 km s\pmOne). The continuum image is created by
averaging over all the line-free channels, this yields a synthesized clean beam size of 16\farcs7 $\times$ 7\farcs2 and an 
rms noise of 0.55 mJy beam\pmOne.

\section{Results}\label{sec:res}
\subsection{New results: Continuum Emission} 
Averaging over all the line-free channels, we detect continuum emission at 10$\sigma$ significance at an averaged frequency of $\nu_{cont}$ = 104.2106 GHz ($\sim$2.9 mm) in the observed-frame, corresponding to 175.6 GHz ($\sim$1.7 mm) at $z$ = 0.685. In this lens system, the 
foreground galaxy (3C220.3) is radio-loud, we thus expect the continuum to be dominated by synchrotron emission from the foreground galaxy. We use CASA's {\sc imfit} task to estimate the peak position of the continuum emission. The deconvolved source size is 8\farcs72 $\pm$ 0\farcs69 $\times$ 3\farcs87 $\pm$ 1\farcs60, and the integrated flux density is 7.18 $\pm$ 0.69 mJy. At the peak position of the continuum emission, the peak flux density is S$_\nu$ = 5.56 $\pm$ 0.54 
mJy beam\pmOne. An overlay image of the 104 GHz continuum emission on the 9 GHz continuum measurement (H14) is shown in Figure \ref{fig:cont}. This demonstrates that at the resolution of our observation, the continuum detection is marginally resolved. It is therefore plausible that non-thermal emission from both the lobes and the radio core dominate the integrated flux density in our measurement.

\begin{figure*}[tbph]
\centering
\includegraphics[width=0.80\textwidth]{Figure/ContPanel}
\caption{The central cross on each image indicates the position of the radio core of 3C220.3. The contour levels of the 104 GHz continuum emission start at $\pm$3$\sigma$, incrementing at steps 
of $\pm$1$\sigma$ of 0.5 mJy beam\pmOne; the contour levels of the 9 GHz continuum 
emission start at $\pm$4$\sigma$ of $\sigma$ = 0.064 mJy beam\pmOne\ and increment at steps of $\pm$2$^n\sigma$, 
where n is a positive integer.
Left: Contour map of the 104 GHz continuum emission in 3C220.3. The beam size is 16\farcs7 $\times$ 7\farcs2, at P.A. = 
-58$\degr$, as indicated in the bottom left corner. Right: Contours of the CARMA 104 GHz continuum emission (red) from the 
foreground radio galaxy 3C220.3 overlaid on the 9 GHz emission (green and grayscale) (H14). The synthesized beam size of the VLA observations is 0$\farcs$6 $\times$ 0$\farcs$2 at P.A. 
76$\degr$. 
\label{fig:cont}}
\end{figure*}

\subsection{New results: \CO line emission}
We detect \CO line emission at 5$\sigma$ significance toward the background SMG SMM J0939 at $z$ = 2.221.
The line profile in Figure \ref{fig:mom0} is extracted at the peak position of the CO emission. Fitting a four-parameter single Gaussian to the spectrum yields a peak flux density of 19.9 $\pm
$ 2.2 mJy superimposed on a continuum level of 3.46 $\pm$ 0.40 mJy, and full width at half-maximum (FWHM) of 542 $\pm$ 32 km s\pmOne. 
The spatial extent of the SMG is $\sim$5$\arcsec$ as shown in the Submillimeter Array (SMA) 1 mm dust continuum in Figure \ref{fig:mom0}. As such, the \CO emission in the SMG is spatially unresolved. We construct a velocity-integrated (0th moment) map of the \CO 
emission using the data with continuum subtracted in the visibility plane. This results in a velocity-integrated \CO line flux of $S_{\rm CO}$ = 10.7 $\pm$ 2.12 Jy km s\pmOne\ over the FWZI velocity range of $\Delta v\sim$914 km s\pmOne, the uncertainty does not include $\sim$ 15\% calibration uncertainty. We ignore primary beam correction given that the source is close to the phase center, and the source is unresolved (\ie much smaller than the primary beam size). This \CO line measurement confirms the redshift of SMM J0939 at $z$ = 2.221 $\pm$ blah.

\begin{figure*}[tbph] 
\centering
\includegraphics[width=0.8\textwidth]{Figure/LinePanel}
\includegraphics[width=0.65\textwidth]{Figure/peak28_WiderSpectrum}
\caption{The central cross on each image corresponds to the same coordinates. The contour levels start at $\pm$2$\sigma$ and $\pm$3$\sigma$ for the CARMA line emission and the SMA continuum emission, respectively, incrementing at
steps of $\pm$1$\sigma$. Top Left: Continuum-subtracted moment-0 map of the \CO line emission toward 
the background SMG with $\sigma$ = 1.8 Jy km s\pmOne\ beam\pmOne over a velocity range of $\Delta v\sim$610 km s\pmOne\ . The angular resolution is 13$\farcs$6 $\times$ 
5\farcs8, as indicated in the bottom left corner. 
Top Right: Contours of the \CO line emission (red) overlaid on the SMA 1 mm dust continuum (green; H14) with $\sigma_{\rm 1 mm}$ = 0.84 mJy beam\pmOne. The beam size of the SMA data is 1\farcs4$ \times $1\farcs2, P.A. -34\degr, as shown 
in the bottom left corner. Bottom: 
The spectrum extracted at the peak position of the CO line emission. The spectral resolution is $\Delta v$ $\sim$29 km s\pmOne\ with an rms of $\sigma_{\rm ch}$ = 12.5 mJy beam\pmOne\ per channel.
Solid black line shows the Gaussian fit to the \CO line profile. Yellow histogram shows the 
flux density as a function of velocity offset, where 0 km s\pmOne\ corresponds to $z$ = 2.221. \label{fig:mom0}}
\end{figure*}


\section{Analysis}
\subsection{Lens Modelling} \label{sec:Lens} 
To study the intrinsic properties of the background galaxy, we determine the magnification factor by performing
lens modeling using the SMA 1 mm archival data of this system. Lens modeling is carried out in the visibility
(uv-) plane using the updated version\footnote{commit: 7aee6276} of the publicly available software {\sc uvmcmcfit}\footnote{https://github.com/sbussmann/uvmcmcfit}
\citep{Bussmann15a}. The code uses an affine-invariant Markov chain Monte Carlo (MCMC) approach to sample the posterior
probability density function (PDF) of the model parameters. In the code, the surface mass densities of both
lenses are represented by singular isothermal ellipsoid (SIE) profiles, and the source is assumed to have an
elliptical Gaussian profile. The code does not include an external shear parameter.

We fix the phase center to coordinates ($\alpha$, $\delta$) (J2000) = (9:39:23.54, 83:15:26.10), the
offset positions of the lenses and source are referenced to this phase center. The primary lens (3C220.3) is
described by 5 free parameters: the angular offset relative to
the chosen phase center in the image ($\Delta \alpha_{\rm
lens0}$ and $\Delta \delta_{\rm lens0}$), the angular Einstein radius ($\theta_{\rm E,0}$), the
axial ratio ($q_{\rm lens0}$), and the position angle ($\phi_{\rm lens0}$). The secondary lens (companion B) is
described by 3 free parameters: $\theta_{\rm E,1}$, $q_{\rm lens1}$, and $\phi_{\rm lens1}$. The angular offset
of the secondary
lens is sampled with respect to ($\Delta \alpha_{\rm lens0}$ and $\Delta \delta_{\rm lens0}$) of
the primary lens.
The source (SMM J0939) is parameterized by
6 free parameters: the position of the source relative to the
primary lens ($\Delta \alpha_{\rm s}$ and $\Delta
\delta_{\rm s}$), the total intrinsic flux density ($S$), the
effective radius ($r_{\rm s} = \sqrt{a_{\rm s} b_{\rm s}}$), the axial
ratio ($q_{\rm s}$ =  $b_{\rm s}/a_{\rm s}$), and the position angle
($\phi_{\rm s}$).
The total number of free parameters is $N_{\rm free}$ = 14. The best-fit model is obtained by maximizing the
Gaussian likelihood function $ \mathcal{L} $ according to:
\begin{equation}
    \mathcal{L} = \sum_{u, v}\left( \frac{|V_{\rm data} - V_{\rm
    model}|^2}{\sigma^2} + {\rm log}(2 \pi \sigma^2) \right)
\end{equation}
\noindent where $\sigma$ is determined from the scatter in the visibilities within a
single spectral window (natural weighting).

We initialize the positions and Einstein radii of both lenses, and the position of the source using the
best-fit values of the lens model H14 performed on Keck K-band (near-infrared) data. For each of
these parameters, we impose a uniform prior in the range $\in\pm$3$\sigma$, where $\sigma$ is the uncertainty
reported in their paper. The axial ratios of the lenses are restricted to $q_{\rm lens} > 0.3$. We initialize 512
walkers and 6000 steps to identify the best-fit model parameters.
\begin{figure}[!tbpH]
\centering
\includegraphics[width=0.232\textwidth]{Figure/LensedSBmap_model_goodfit1970}
\includegraphics[width=0.232\textwidth]{Figure/LensedSBmap_residual_goodfit1970}
\caption{Double-lens modeling of SMM J0939 using {\sc uvmcmcfit} on the SMA 1 mm continuum data.
The contours start at $\pm$2$\sigma$, incrementing at
steps of $\pm$2$\sqrt{\rm 2}\sigma$. Solid contours show the positive residuals and dashed contours
show the negative residuals. 
Left: SMA 1 mm continuum (red contours) overlaid on the best-fit model (grayscale image) assuming an elliptical Gaussian profile for the background SMG. The lenses are represented as black dots, the half-light area of the background source is represented as magenta ellipse, and the critical curves are represented as orange curves. 
Right: Residual contours and image obtained by taking the Fourier transform of the difference between the SMA data and the best-fit model in the visibility plane. \label{fig:lens}}
\end{figure}

The resulting best-fit model as shown in Figure \ref{fig:lens} shows no significant bowls in the residual
image, and the knots (lensed emission) in the observed SMA data are well-reassembled with the best-fit model.
The best-fit model yields a magnification
factor of $\mu_{\rm L}$ = 10.13 $\pm$ 1.38, this is consistent the value reported by H14 within errors. All best-fit
parameters are listed in Table \ref{tab:lensParam}. The Einstein radii associated with the best-fit model for the two lenses are $\theta_{E}$ =
1.22 $\pm$ 0.01 (8.75 kpc at $z$ = 0.685) and $\theta_{E}$ = 0.75 $\pm$ 0.02 (5.34 kpc at $z$ = 0.685),
the corresponding masses within the Einstein radii are $M(\theta < \theta_{\rm E})$ = (4.86 $\pm$ 0.08) $\times$ 10$^{11}$ 
\Msun\ 
and $M(\theta < \theta_{\rm E})$ = (1.82 $\pm$ 0.07) $\times$ 10$^{11}$ \Msun, respectively. 
\begin{deluxetable}{l l r}[tbpH]
\tabletypesize{\scriptsize}
\tablecolumns{3}
\tablewidth{0pc}
\tablecaption{Lens modeling results}
\tablehead{
\multicolumn{2}{c}{Parameters} &
\colhead{Best-Fit Values}
\\ \cline{1-3} \vspace{-0.05in} \\
% \tableline
\multicolumn{3}{c}{Lens 0}
}
\startdata
%\cutinhead{Lens 0}
$\Delta \alpha_{\rm lens,0}$      & (\arcsec)   & 0.377 $\pm$ 0.026     \\
$\Delta \delta_{\rm lens,0}$      & (\arcsec)   & -0.209 $\pm$ 0.027    \\
$q_{\rm lens0}$ \tablenotemark{a} &             & 0.446 $\pm$ 0.063     \\
$\phi_{\rm lens0}$                & (deg)       & 33.22 $\pm$ 4.15\phn  \\
$\theta_{\rm E,0}$                & (\arcsec)   & 1.223 $\pm$ 0.010     \\
\cutinhead{Lens 1}
$\Delta \alpha_{\rm lens1}$       & (\arcsec)   & -0.788 $\pm$ 0.034    \\
$\Delta \delta_{\rm lens1}$       & (\arcsec)   & -1.261 $\pm$ 0.017    \\
$q_{\rm lens1}$ \tablenotemark{a} &             & 0.529 $\pm$ 0.138     \\
$\phi_{\rm lens1}$                & (deg)       & 9.55 $\pm$ 15.70      \\
$\theta_{\rm E,1}$                & (\arcsec)   & 0.733 $\pm$ 0.015     \\
\cutinhead{Source}
$\Delta \alpha_{\rm s}$           & (\arcsec)   & -0.120 $\pm$  0.035   \\
$\Delta \delta_{\rm s}$           & (\arcsec)   & -0.148 $\pm$  0.048   \\
$q_{\rm s}$ \tablenotemark{a}     &             & 0.012 $\pm$ 0.237     \\
$\phi_{\rm s}$                    & (deg)       & 175.35 $\pm$ 8.89\phn \\
$r$\tablenotemark{b}              & ($\arcsec$) & BLAH0.024 $\pm$   0.033   \\
$\mu$                             &             & 9.74 $\pm$ 1.38\phn
\enddata
% 0.377 & -0.209 & 0.446 & 33.22 & 1.223 & -0.788 & -1.26 & 0.5289 & 9.55 & 0.733 & 9.74
\label{tab:lensParam}
\tablenotetext{a}{Axial ratio}
\tablenotetext{b}{Effective Radius}
% \tablenotetext{c}{}
\tablecomments{The angular offsets listed above are with respect to $\alpha$ = 144.84809, $\delta$ = 83.25725 (J2000). }
\end{deluxetable}

















\subsection{SED fitting} \label{sec:SED}
\subsubsection{3C220.3}
Synchrotron continuum emission from extended components of a radio galaxy decreases with increasing frequencies, 
and the spectrum is commonly characterized by a power law distribution $S \propto \nu^{-\alpha}$ where the 
spectral index $\alpha$ is $\gtrsim$ 0.5. While the contribution from extended components decreases, studies using 
samples of radio galaxies have suggested that the flat/inverted-spectrum of the compact radio core component rises 
and dominates the flux density at higher frequencies \citep{Kellermann81a,Begelman84a}. At observed-frame $\sim$ 90 GHz, 
observations of the FR-II radio galaxy, 3C220.1 at $z$ = 0.610 has been reported to be the case 
\citep{Hardcastle08a}. Besides, an upper 
limit of $<$ 0.17 mJy at 4.6 GHz \citep{Mullin06a} and a clear detection of 0.8 mJy at 9 GHz (H14) appears to 
be indicative of a substantially inverted spectrum of the core (Figure \ref{fig:SED}).
Consequently, we would naively expect the integrated flux density in our continuum detection of $S_{\rm 104GHz}$ = 7.18 $\pm$ 0.69 mJy to be dominated by the unresolved core component of the foreground FR-II, which is at $z$ = 0.685. == wouldn't this be contradicting if we know the core is unresolved, why would we jump into looking at the integrated flux ==
However, the inferred deconvolved spatial size of the source matching that in a resolved image (in Figure 1) is 
suggestive of a marginally resolved detection of the extended components with non-negligible emission. 
This is plausible given that the orientation of the synthesized beam in our observations is in alignment with the 
axis along the 
lobes of the radio galaxy, as shown in Figure \ref{fig:cont}. We investigate this disparity by fitting an SED to 
existing data of the total integrated flux as listed in Table \ref{tab:SEDdataRadio}, and extrapolating the fit to 
estimate the flux density of the lobes at the frequency of our continuum measurement. 
\begin{deluxetable}{rlrcc}[tbpH]
\tabletypesize{\scriptsize}
\tablecolumns{5}
\tablecaption{Continuum data of 3C220.3}
\tablehead{
\multicolumn{2}{c}{Frequency} &
\multicolumn{2}{c}{Flux Density} &
\colhead{Ref.\tablenotemark{b}}
%\\ \tableline
\\ \cline{1-5}  \vspace{-0.05in} \\
\multicolumn{5}{c}{3C220.3}
}
\startdata
    104.2 & GHz & 7.18 $\pm$ 0.051                      & mJy & This work \\
    10.7  & GHz & 270 $\pm$ 30                          & mJy & K73       \\
    10.7  & GHz & 253 $\pm$ 28                          & mJy & L80       \\
    9.0   & GHz & 0.80  $\pm$    0.06  \tablenotemark{a} & mJy & H14       \\
    5.0   & GHz & 640 $\pm$ 100                         & mJy & K69       \\
    5.0  & GHz & 636 $\pm$ 50                          & mJy & L80   \\
    4.86 & GHz & $<$ 0.17  \tablenotemark{a} & mJy & M06   \\
    2.7  & GHz & 1.33 $\pm$ 0.07                       & Jy & K69   \\
    2.7  & GHz & 1.34 $\pm$ 0.10                     & Jy & L80   \\
    1.4  & GHz & 2.95 $\pm$ 0.09                      & Jy & C98   \\
    1.4  & GHz & 2.99 $\pm$ 0.06                      & Jy & P66   \\
    1.4  & GHz & 2.80 $\pm$ 0.14                      & Jy & K69   \\
    1.4  & GHz & 2.89 $\pm$ 0.09                      & Jy & L80   \\
    0.75 & GHz & 5.94 $\pm$ 0.28                      & Jy & L80   \\
    0.75 & GHz & 5.94 $\pm$ 0.21                      & Jy & P66   \\
    0.75 & GHz & 5.60 $\pm$ 0.84                      & Jy & K69   \\
    352  & MHz & 11.3 $\pm$ 0.453                      & Jy & WENSS \\
    352  & MHz & 11.6 $\pm$ 0.464                      & Jy & WENSS \\
    178  & MHz & 15.7 $\pm$ 2.35                      & Jy & K69   \\
    178  & MHz & 17.1 $\pm$ 1.71                      & Jy & L80   \\
    152  & MHz & 22.6 $\pm$ 0.08                       & Jy & B85   \\
    152  & MHz & 22.5 $\pm$ 0.04                      & Jy & B85   \\
    86   & MHz & 51.6 $\pm$ 9.90                       & Jy & L80   \\
    73.8 & MHz & 37.5 $\pm$ 3.82                       & Jy & C07   \\
    38   & MHz & 49.6 $\pm$ 4.96                       & Jy & L80   \\
    38   & MHz & 40.2 $\pm$ 6.30                       & Jy & K69   \\
    37.8 & MHz & 60.7 $\pm$ 6.07                       & Jy & H95   \\
    17.8 & MHz & 64.9 $\pm$ 6.49                       & Jy & H95   \\
% # Our Data: [peak, integrated, difference]
% Point_Continuum_Jy = np.array([5.56e-3, 7.18e-3, 1.62e-3])
% Point_error_Jy = np.array([5.0647e-4, 5.0647e-4, 5.0647e-4])    # Jy
\enddata
\label{tab:SEDdataRadio}
\tablenotetext{a}{Core}
\tablenotetext{b}{References.~
K93 = \citet{r9};
L80 = \citet{r10-13-15-19-20-26-29-31};
K69 = \citet{r11-14-18-22-25-32};
H14 = \citet{Haas14};
M06 = \citet{Mullin06a};
C98 = \citet{r16};
P66 = \citet{r17-21};
B85 = \citet{r2728};
C07 = \citet{r30};
H95 = \citet{r33-34};
WENSS = \citet{r23-24}\footnote{http://www.astron.nl/wow/testcode.php\?survey=1};
}
% \tablenotetext{c}{}
\tablecomments{}
\end{deluxetable}
















Following Equation (1) in \citet{Cleary07a}, the fit to the lobes emission can be expressed as a parabolic function:
\begin{equation}
\log F_{\nu}^{\mathrm lobe} (\nu) \propto - \beta\ (\log\ \nu - \log \nu_{t})^2  + \log (\exp({\frac{\nu}{\nu_c^{\mathrm lobe}}}))
\end{equation}
where $F_{\nu}^{\mathrm lobe}$ is the flux density of the lobes, $\beta$ is a parameter representing the bending 
of the parabola, $\nu_t$ is the frequency at which the optical depth of the synchrotron emitting plasma reaches 
unity, and $\nu_c^{\rm lobe}$ is the frequency corresponding to the cutoff energy of the lobe plasma energy 
distribution. 
The extrapolated flux density at 104 GHz is consistent with the peak flux density of our continuum 
measurement (Figure \ref{fig:SED}). The 10$\sigma$ detection of the continuum thereby suggests
a dominant contribution from the lobes, and the peak flux density therefore does not appear to be emission toward 
the core. Moreover, the peak position of the 104 GHz continuum is
centered toward the northern lobe (Figure \ref{fig:cont}), which further supports our argument. We did not 
extrapolate the core measurements to the frequency of our continuum, as previous measurements of the core are 
taken 
across epochs.\par
Studies by \citet{Meisenheimer89a} and \citet{Hardcastle08a} have suggested spectra of hotspots are flat up to optical frequencies, and some exhibit spectral steepening either between cm or mm wavelengths (\eg 3C123); at the resolution of our observation, however, it is unclear whether the integrated flux is dominated by the emission from the compact hotspot or the emission from the surrounding diffuse lobe components. Nonetheless, high resolution observations of 3C220.3 at mm wavelengths (rest-frame $\sim$ 60 GHz) will provide important insight on hotspot emission studies. 

\begin{figure}[!tbph]
\centering
\includegraphics[width=0.5\textwidth]{Figure/3C220_3_FullSED_longticks.pdf}
\caption{The SED of 3C220.3 (solid blue line) and SMM J0939 (dashed blue line and solid cyan line) including the new measurements presented in this paper. 
Black dots represent existing data of 3C220.3 (see Table \ref{tab:SEDdataRadio}). Red dots at 104 GHz correspond to 
our continuum measurements (integrated, peak, and difference). The purple line corresponds to the parabolic function we 
fit to the data associated with the radio galaxy, following \citet{Cleary07a}. The dashed purple line and 
the solid cyan line correspond to the best-fit SED models of the background SMG. The photometric data of the SMG are reported by H14. \label{fig:SED}}
\end{figure}

\subsubsection{SMM J0939+8315} 
To constrain the dust and gas properties in the ISM of SMM J0939, we perform SED fitting on the
photometric data obtained with {\it Herschel}/PACS and SPIRE, at wavelengths
between observed-frame 70 \micron $-$ 500 \micron, and interferometric data with the SMA at 1 mm (H14). We use the publicly
available software {\sc mbb\_emcee}\footnote{https://github.com/aconley/mbb\_emcee} to perform SED fitting, the code uses an affine-invariant Markov chain Monte
Carlo (MCMC) approach, the details are described by \citet{Riechers13a} and \citet{Dowell14a}. The
functional form of the fit comprises a single-temperature, modified blackbody function joined to a $S_{\lambda} \propto \lambda^\alpha
$ power law on the blue
side of the SED.
We fit both optically thick and optically thin models. In the optically thick case, the wavelength $
\lambda_0$ = ${c}/{\nu_0}$ is an additional parameter which represents where the optical
depth $\tau_{\nu} =$ ($\nu$/$\nu_0$)$^\beta$ reaches unity. Thus, the functional form of the modified blackbody
in the optically thick regime is as follows:
\begin{equation}
\rm S_{\lambda} \propto \frac{(1-exp^{-(\frac{\lambda_0 (1+z)}{\lambda})^{\beta}})(\frac{c}{\lambda})^3}
{exp^{\frac{hc}{\lambda\rm{kT/(1+z)} } }-1}
\end{equation}
and in the optically thin regime, the functional form reduces to:
\begin{equation}
\rm S_{\lambda} \propto \frac{(\frac{c}{\lambda})^{\beta+3}}{exp^{\frac{hc}{\lambda\rm{kT/(1+z)}}}-1}
\end{equation}
where $T$ is the rest-frame characteristic cold dust temperature, $\lambda_0$ is rest-frame wavelength
where the optical depth reaches unity, $\beta$ is the dust emissivity (or spectral index of the dust extinction
curve), and $\alpha$ is the power law spectral index. The overall fit is normalized using the observed-frame 500
$\micron$ flux density, hence this becomes an additional parameter in the fit. In both models, we impose an upper limit on $
\lambda_0$ to 2000 \micron, and on the observed-frame dust temperature $T/(1+z)$ to 60 K. We fix the upper limit on 
$\beta$ to be 3.0 and 2.2 for the optically thin model and optically thick model, respectively.

\begin{deluxetable}{ccc}[tbpH]
\tabletypesize{\scriptsize}
\tablecolumns{3}
\tablecaption{SED fitting results}
\tablehead{
\colhead{Parameters}                  &
\colhead{Optically Thick}       &
\colhead{Optically Thin}
}
\startdata
$\chi^2$ & 2.25 & 5.31 \\
D.O.F & 2 & 3 \\
$T_{\rm d}$ (K) & 60.91$^{+1.08}_{-1.31}$ & 51.95$^{+1.26}_{-1.21}$ \\
$\beta$ & 1.35$^{+0.57}_{-0.53}$ & 0.7$^{+0.24}_{-0.26}$ \\
$\alpha$ & 3.05$^{+0.31}_{-0.40}$ & 2.76$^{+0.23}_{-0.23}$ \\
%$\lambda_0$(1+$z$) ($\micron$) \tablenotemark{c} & 722.81$^{+276.88}_{-398.67}$ & --- \\
$\lambda_0$ ($\micron$) \tablenotemark{e} & 224.41$^{+85.96}_{-123.77}$ & --- \\
$\lambda_{\rm peak}$ \tablenotemark{c}\micron & 254.7$^{+6.2}_{-6.1}$ & 301.4$^{+29.0}_{-30.1}$ \\
$f_{\rm norm, 500 \micron}$ mJy  \tablenotemark{c} & 255.79$^{+16.67}_{-16.31}$ & 244.25$^{15.28}_{15.30}$ \\
$L_{\rm (8-1000)\micron}$ [10$^{12}$ L$_\sun$] \tablenotemark{d} & 88.52$^{+2.62}_{-2.63}$ & 89.15 \\
$M_{\rm d}$ [10$^8$ M$_\odot$] \tablenotemark{b} & 50.47$^{+20.42}_{-20.15}$ & 25.74$^{+3.88}_{-5.49}$
\enddata

% Optically thick & 18.91 & 1.35 & 722.81 & 3.05 & 254.7 & 88.52 & 50.47 \\
% Optically thin &  16.13 & 0.7 & N/A   & 2.76 & 301.4 & 89.15 & 25.74

\label{tab:mbb}
\tablenotetext{a}{The observed-frame wavelength where the dust becomes optically thick}
\tablenotetext{b}{Assuming standard absorption mass coefficient $\kappa$=2.64 m$^2$ kg$^{-1}$ at $\lambda$=125.0 $\micron$ (Dunne et al. 2003), without lensing correction}
\tablenotetext{c}{observed-frame}
\tablenotetext{d}{rest-frame 8-1000 $\micron$ without lensing correction}
\tablenotetext{e}{The rest-frame wavelength where the dust becomes optically thick, upper limit is 2000 $\micron$}
\tablecomments{Errors are $\pm$1$\sigma$}
\end{deluxetable}

% thick_500_500.log
% T/(1+z): 18.91 +1.09 -1.31 (low lim: 1.00 upper lim: 60.00) [K]
% beta: 1.35 +0.57 -0.53 (low lim: 0.10 upper lim: 2.20)
% fnorm: 255.79 +16.67 -16.31 (low lim: 0.03) [mJy]
% lambda0 (1+z): 722.81 +276.88 -398.67 (low lim: 1.00 upper lim: 3049.15) [um]
% alpha: 3.05 +0.31 -0.40 (low lim: 0.10 upper lim: 20.00)
% Lambda peak: 254.7 +6.2 -6.1 [um]
% L_IR(8.0 to 1000.0um): 88.52 +2.62 -2.63 [10^12 L_sun]
% M_d(kappa=2.64, lam=125.0um): 50.47 +20.42 -20.15 [10^8 M_sun]
% Number of data points: 7
% ChiSquare of best fit point: 2.25

% note using beta upper limit 3.0, getting very different beta, and dust mass
% Fit results:
% T/(1+z): 19.75 +0.56 -0.53 (low lim: 1.00 upper lim: 60.00) [K]
% beta: 1.91 +0.76 -0.76 (low lim: 0.10 upper lim: 3.00)
% fnorm: 267.13 +16.03 -15.98 (low lim: 0.03) [mJy]
% lambda0 (1+z): 1012.99 +142.79 -275.29 (low lim: 1.00 upper lim: 3049.15) [um]
% alpha: 3.64 +0.08 -0.86 (low lim: 0.10 upper lim: 20.00)
% Lambda peak: 255.6 +6.3 -6.2 [um]
% L_IR(8.0 to 1000.0um): 88.02 +2.85 -2.85 [10^12 L_sun]
% M_d(kappa=2.64, lam=125.0um): 108.23 +32.00 -63.67 [10^8 M_sun]
% Number of data points: 7
% ChiSquare of best fit point: 2.25
% Saving results to thick_testbeta.h5

% thin_testSMA
% T/(1+z): 16.13 +1.26 -1.21 (low lim: 1.00 upper lim: 60.00) [K]
% beta: 0.70 +0.24 -0.26 (low lim: 0.10 upper lim: 3.00)
% fnorm: 244.25 +15.28 -15.30 (low lim: 0.03) [mJy]
% alpha: 2.76 +0.23 -0.23 (low lim: 0.10 upper lim: 20.00)
% Lambda peak: 301.4 +29.0 -30.1 [um]
% L_IR(8.0 to 1000.0um): 89.15 +2.48 -2.51 [10^12 L_sun]
% M_d(kappa=2.64, lam=125.0um): 25.74 +3.88 -5.49 [10^8 M_sun]
% Number of data points: 7
% ChiSquare of best fit point: 5.31

The best-fit values in both regimes are listed in Table \ref{tab:mbb}, and the correlation plots are available in the Appendix. The best-fit solution of an optically thin
model corresponds to $\chi^2$ = 5.31 with 3 degrees of freedom, whereas that of an optically thick model
corresponds to $\chi^2$ = 2.25 with 2 degrees of freedom, suggesting a better fit than in the optically thin
case. In the subsequent analysis, we employ the inferred values from the optically thick model.
The best-fit solution yields a far-infrared (FIR) luminosity of $L_{\rm FIR (42.5-122.5\micron)}$ = 53.3$^{+1.1}_{-1.1}\times$10$^{12}$\Lsun, and a total infrared (IR) luminosity of $L_{\rm IR (8-1000 \micron)}$ = 88.5$^{+2.6}_{-2.6}\times$10$
^{12}$\Lsun. Assuming a dust absorption coefficient of $\kappa_{\nu}$ = 2.64 m$^2$ kg\pmOne\ at 125.0 $
\micron$ \citep{Dunne03a}, we derive the dust mass using the following expression:
\begin{equation}
M_{\rm dust} = S_{\nu} D_{L}^2 [(1 + z) \kappa_{\nu} B_{\nu}]^{-1} \tau_{\nu} [1-
\exp(-\tau_{\nu})]^{-1}
\end{equation}
where $S_{\nu}$ is the flux density in ergs/s/cm$^{2}$/Hz , $D_{\rm L}$ is the luminosity distance in cm, $\kappa_{\nu}$ is the dust
absorption coefficient in cm$^2$ g\pmOne, $\tau_{\nu}$ is the optical depth, and $B_{\nu}$ is the Planck function,
all quantities are expressed in the observed-frame. We find a dust mass of $M_{\rm dust}$ =
50.5$^{20.4}_{-20.2}\times$10$^8$\Msun, the uncertainties do not include those of $\kappa_{\nu}$. These values are based on SED fitting to the photometric data, \ie prior
to lensing correction. 

With the limited amount of data in the FIR waveband, the dust mass is weakly constrained. 
We perform an additional optically thick model fitting where we loosen the upper limit of $\beta$ from 2.2 to 3.0, while the difference between all best-fit parameters in this scenario and those using an upper limit of 2.2 (see Table \ref{tab:mbb}) is within 3\%, we find that the dust mass is boosted by a factor of $\sim$2. 

\subsection{Molecular gas mass}
While the ground state CO transition line, CO($J$ = 1 \rarr\ 0) traces the cold molecular gas in the ISM
\citep*[\eg][]{Downes98a,Wilson70a}, CO($J$ $>$ 1) transition lines are frequently observed at high redshifts as the
 ground state CO transition line is redshifted to lower frequencies that can only be observed with a few telescopes 
 \citep{Carilli13a}, 
 hence assumptions on the CO excitation conditions are required to derive the molecular gas mass using the H$_{\rm 2}$-to-CO 
 relation. 
Prior to the observations of the ground state CO line in SMGs, it has been assumed that the molecular gas in the
  ISM traced by CO lines is thermalized due to their high star formation rates \citep[\eg][]{Greve05a, Coppin08a}.
   Yet, recent observations of this line have demonstrated that SMGs can indeed be subthermally excited
   \citep{Harris10a,Riechers11c,Riechers11d,Ivison11a}; based on observations toward SMGs of both the \CO and the CO($J$ = 1 \rarr\ 0) lines, the 
   brightness temperature ratio has been inferred as $R_{\rm 31}<$ 0.8 \citep
   {Harris10a,Carilli10a,Swinbank2010a,Riechers11d,Ivison11a,Ivison10d}. In contrast, observations of high-redshift quasar hosts suggest that the ratio 
   is $R_{\rm 31}\sim$ 1 \citep{Riechers06a, Riechers11a, Scott11a}. 
Here, we derive the molecular gas mass assuming thermalized excitation of CO, \eg we adopt $R_{\rm 31}$ = 1 as SMM J0939 is 
postulated to be hosting a type-2 quasar. \par
We calculate the CO($J$ = 1 \rarr\ 0) line luminosity using standard relation 
\citep[\eg][]{Solomon05a,Carilli13a}:
\begin{equation}
L^{\prime}_{\rm CO} = \frac{3.25\times10^7}{\nu_{\rm CO (3-2), rest}^2}\times \frac{D_L^2}{\mu} \times
\frac{I_{\rm CO(3-2)}} {R_{\rm 31} (1 + z)}
\end{equation}
where $\nu_{\rm CO (3-2), rest}$ is the rest-frame frequency of the \CO line in GHz, $D_L$ is the luminosity distance in Mpc, and $I_{\rm CO(3-2)}$ is the \CO line flux in Jy km s\pmOne. This corresponds to $L^{\prime}_{\rm CO (1-0)}$ = (2.91 $\pm$ 0.78)$\times$10$^{10}$(10.1/$\mu_{\rm L}$) \LpU\ after correcting for lensing magnification; the inferred total molecular gas mass is therefore $M_{\rm gas}$ = (2.33 $\pm$ 0.62) $\times$10$^{10}$\Msun. We assumed a conversion factor of $\alpha_{\rm CO}$ =
0.8 \Msun (\LpU)\pmOne\ based on empirical relations from local ULIRGs, which is typically
adopted for SMGs \citep[\eg][]{Tacconi06a,Tacconi08a,Bothwell13a}. 
This results in a gas-to-dust
ratio of $f_{\rm gas-dust}$ = $M_{\rm gas}/M_{\rm dust}$ = 47 $\pm$ 21, this is in good agreement with the 
values found in other SMGs \citep{Coppin08a,Micha10a,Riechers11c}.
% find an excellent agreement of blah with blah
\subsection{Star formation rate}
We derive the star formation rate (SFR) using the \citet{Kennicutt98a} relation, assuming a \citet{Chabrier03a}
stellar initial mass (IMF) function. This yields a SFR$_{\rm IR}$ of 874 $\pm$ 122 $M\sun$ yr\pmOne, derived using the lensing-corrected total IR luminosity (L$_{\rm IR}$) derived from SED 
fitting, and assuming negligible contributions from AGN heating, which is at most BLAH \% as no evidence of warm dust (from MIR). \footnote{SFR$_{\rm FIR}$ = 526 $\pm$ 72 $M_
\odot$ yr\pmOne\ }

Assuming constant SFR, the minimum time for which the starburst in SMM J0939 can be maintained at its
current SFR can be approximated as the gas depletion timescale, $\tau$ = $M_{\rm gas}$/SFR. 
This corresponds to $\tau_{\rm IR}$ = 25.6 $\pm$ 0.6 Myr.

Move to discussion: Compare THIS NUMBER: with other SMGs \citep[\eg][]{Greve05a}. 

\subsection{Star formation efficiency}
The SFR per unit mass of molecular gas is commonly taken as a
measure of the star formation efficiency, SFE = SFR/$M_{\rm gas}$. We compute this ratio using lensing-corrected infrared 
luminosities ($L_{\rm IR}$) and CO($J$ = 1 \rarr\ 0) luminosity: SFE = $L_{\rm IR}$/\Lp, this ratio makes no assumptions on the gas mass conversion factor $\alpha$ as well as the 
choice of IMF, however, this assumes that differential lensing effect between the CO and infrared emission is negligible. 
The resulting ratios are SFE$_{\rm IR}$ = 300 $\pm$ 10 Myr\pmOne, which is comparable
to what is found in "typical" SMGs \citep{Riechers10, Tacconi06, Greve05a}.

\subsection{Dynamical mass} 
We derive the dynamical mass of SMM J0939 using an isotropic virial estimator \citep[\eg][]{Engel10a}:
\begin{equation}
M_{\rm dyn} = 2.8\times 10^5\Delta v_{\rm line}^ 2 R_{\rm eff}
\end{equation}
where $\Delta v_{\rm line}$ is in km s\pmOne, $R_{\rm eff}$ is in kpc, and $M_{\rm dyn}$ is in \Msun.
We employ the FWHM line width of our \CO measurement for $\Delta v_{\rm line}$,
and the half-light radius from lens modeling for $R_{\rm eff}$, assuming the dust emission traces the same emitting region as the CO. This yields $M_{\rm dyn}$ = (7.4 $\pm$ 2.4) $\times$ 10$^{10}$\Msun , and gas-to-dynamical mass fraction of $f_{\rm gas-to-dyn}$ = $\approx$ 0.32, consistent that of other SMGs \citep{Greve06a,Tacconi06a}.

\subsection{SFR and gas surface densities}
To compute surface densities we divided half of the star formation rate or gas mass (so that no assumption of IMF) by the area subtended by the half-light (or effective) radius.
 the surface density of star formation against the molecular gas surface density. 


\section{Discussion And Conclusions} \label{sec:conclusions}
We present blah

This paper focuses on a blah source J0939+8315, a lensed SMG that is likely to be hosting a type-2 quasar, this source offers an exceptional opportunity to
investigate the gas mass and dynamics of this
poorly-studied population of high z galaxies.

In this paper, we present blah of a strongly lensed blah at z = 2.221, at the epoch of cosmic star formation (cite)

We report a detection of \CO line.
The continuum detection is marginally resolved, we place constraints on the SED of the foreground galaxy (accounting for radio 
core and lobes emission). Our CO measurement confirms the redshift of this source, which was previously measured using CIV and HeII lines (H14).

The detection of CO in SMM J0939, implying a large molecular gas reservoir, demonstrates the active SF in SMGs. 
The large CO luminosity suggests that a signification fraction of the FIR luminosity comes from SF, providing power for dust to re-radiate in FIR. BUt we also know that this source has an dust-enshrouded AGN, boosting the IR.
The SFR inferred from the CO measurement is common for SMGs. the large amount of molecular gas further supports the picture that SMGs are galaxies with making up most of the stellar mass in the early universe, and is responsible for galaxy evolution and formation. 

The magnification factor is blah, providing an opportunity to carry out detailed studies of the ISM in this population.

The properties of this SMG blah, M$_{\rm dust}$, L$_{\rm FIR}$ of blah, L$_{\rm IR}$ of blah, yielding SFR of 
blah . Comparing our findings to other SMGs / similar populations, we find blah, which is blah.

We derive physical properties of the molecular gas in the ISM using IR as well as FIR luminosities. As studies have shown that contaminations of AGN in the FIR is less than blah \%, while most studies on SMGs uses the bolometric IR luminsoity, hence we derive these quantities using LIR as well for comparison. 

We compare the properties of SMM J0939 with two other SMGs with detailed studies -- HLSW-01 and the cosmic Eyelash. All sources are lensed, making spectral lines measurements relatively feasible than the fainter, un-lensed sources of the same population.

Properties and comparison with similar galaxies in Table \ref{tab:comapreSMG}
\newcommand\tnh{\,\tablenotemark{h}}
\newcommand\tni{\,\tablenotemark{i}}
\newcommand\tnj{\,\tablenotemark{j}}
\newcommand\tnk{\,\tablenotemark{k}}
\begin{deluxetable*}{l l c c c c c}[tbpH]
\tabletypesize{\scriptsize}
\tablecolumns{7}
\tablecaption{Comparison of SMM J0939 with SMGs at $z\sim$2}
\tablehead{
\multicolumn{2}{c}{SMGs}       &
\colhead{SMM J0939}  &
\multicolumn{2}{c}{HLSW-01}    &  
\multicolumn{2}{c}{Cosmic Eyelash} 
                     \\
\colhead{Quantity} &
\colhead{Unit} &
\colhead{}                     &
\colhead{}                     &
\colhead{Ref.}                     &
\colhead{}                     &
\colhead{Ref.}
}
\startdata
$z$             &                   & 2.221            & 2.957            & R11              & 2.326          &  S10 \\
$\mu_{\rm L}$         &                   & 10.1 $\pm$ 1.4    & 10.9 $\pm$ 0.7 & G11              & 37.5$\pm$4.5    &  S11 \\
$S_{\rm 250}$ & mJy & 440 $\pm$15 (H14) & 420 $\pm$ 10  & R11              & 366 $\pm$ 55  & I10             \\
$I$\tnb       & Jy km s$^{-1}$   & 10.7 $\pm$ 2.1   & 9.7 $\pm$ 0.5  & R11              & 13.2 $\pm$ 0.1 &  D11 \\
$\Delta v_{\rm FWHM}$\tnb & km s$^{-1}$ & 542 $\pm$ 32 & 350 $\pm$ 25 & R11 & $\lesssim$ 800\tnd & D11 \\
\Lp & 10$^{10}$ \LpU & 2.91 $\pm$ 0.78\tnh & 4.17 $\pm$ 0.41 & R11 & 1.7 $\pm$ 0.2 & D11 \\
$M_{\rm gas}$ & 10$^{10}$ \Msun & 2.33 $\pm$ 0.62\tnh & 3.3 $\pm$ 0.3 & R11 & 1.6 $\pm$ 0.1 & I10 \\
$L_{\rm IR}$ &  10$^{12}$ \Lsun & 9.1 $\pm$ 1.2\tnh & 14.3 $\pm$ 0.9 & C11 & 2.3 $\pm$ 0.2 & I10 \\
$M_{\rm dust}$ & 10$^8$ \Msun & 5.2 $\pm$ 2.1\tnh  & 1 - 5.2
& R11 & $\sim$4.0 & I10 \\
SFR$_{\rm IR}$\tna & \Msun yr$^{-1}$ & 874 $\pm$ 122\tnh & 1430 $\pm$ 160\tnc & C11 & $\sim$235\tnc & I10 \\
$\tau_{\rm depl}$\tng & Myr & 25.6 $\pm$ 0.6 & 23 $\pm$ 3\tnc  & R11 & 68\tne & --- \\
$f_{\rm gas-dust}$\tng &  & 47 $\pm$ 21 & 60-330 & R11 & $\sim$ 40 & I10 \\
SFE\tng  & Myr$^{-1}$ & 300 $\pm$ 10 & 340 $\pm$ 40 & R11 & 165 $\pm$ 7 & D11 \\
$M_{\rm dyn}$\tng & 10$^{10}$ \Msun & 7.4 $\pm$ 2.4 & 3.7 $\pm$ 1.8\tne\tnf & --- & 6.0$\pm$0.5 & S11 \\
$f_{\rm gas-dyn}$ && 0.32$\pm$0.14 & 0.90\tne\tnf & --- & 0.6$\pm$0.1 & S11 \\
\enddata
\label{tab:comapreSMG}
\tablenotetext{a}{Chabrier IMF}
\tablenotetext{b}{\CO}
\tablenotetext{c}{Converted theirs values derived using Salpeter IMF to Chabrier IMF}
\tablenotetext{d}{Estimated from Figure 1 in D11}
\tablenotetext{e}{We derive this using the reported values}
\tablenotetext{f}{Using CO($J$ = 5 \rarr\ 4)}
\tablenotetext{g}{Independent of lensing magnification factor $\mu_{\rm L}$}
\tablenotetext{h}{Errors include uncertainties on $\mu_{\rm L}$}
%\tablenotetext{i}{Based on 1 mm continuum}
%\tablenotetext{j}{Based on }
%\tablenotetext{k}{Based on 870 micron, CO 1-0, 6-5, and HST}
%\tablenotetext{f}{Using $L_{\rm IR} / M_{\rm gas}$}
\tablecomments{Values listed from rows 6 onwards are lensing-corrected. References.~
C11 = \citet{Conley11a};
D11 = \citet{Danielson11a};
G11 = \citet{Gavazzi11a};
I10 = \citet{Ivison10c};
R11 = \citet{Riechers11b};
S11 = \citet{Swinbank11a};
S10 = \citet{Swinbank2010a}
}
\end{deluxetable*}
















Discussion of SMG, broad, population.
look at "Lensed QSOs at high z observational, CO measurements (DB)" note

In this paper, we blah. However, many questions regarding jets in radio galaxies remain. For example,

Comparison to CO in other type-2 QSO: F10214+2724 \citep{Riechers11a}

\acknowledgments

We acknowledge the WENSS team for providing the radio measurements for 3C220.3.
Facilities: \facility{CARMA}

\bibliographystyle{apj}
\bibliography{J0939}

\appendix

\begin{figure}[!tbp]
\centering
\includegraphics[width=0.90\textwidth]{Figure/posteriorPDFs}
\caption{Marginalized posterior probabilities of each parameter in our lens modeling.
\label{fig:LensPDF}}
\end{figure}

\begin{figure}[!tbp]
\centering
\includegraphics[width=0.47\textwidth]{Figure/CorrelationPlot__thick_500_500}
\includegraphics[width=0.47\textwidth]{Figure/CorrelationPlot__thin_testSMA}
\caption{Correlation plots from our SED fitting. Plots on the diagonal axes are marginalized posterior probability 
distribution of each
parameter. Off-diagonal plots are 2D histograms between parameters. Crosses denote the best-fit values in 
the 2D correlation plot. Top: Optically thick
model. Bottom: Optically thin model. $T/(1+z)$ is the observed-frame characteristic dust temperature.
\label{fig:sedlikelihood}}
\end{figure}



\end{document}

